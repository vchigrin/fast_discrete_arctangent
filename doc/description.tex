\documentclass{article}
\usepackage{amsmath}
\begin{document}
\title{Fast discrete arctangent computation}
\author{Vyacheslav Chigrin}
\maketitle

\section{Introducion}
Discrete arctangent in that article is a two argument function $\mathcal{F}(x, y)$ parametrized by
integral parameter N. Current library supports only $N = 8k, k \in Z, N > 8$.
It returns zero-based "sector number" in which lies beam from coordinate origin to
point (x, y). Sectors are counted from 0 to N-1 counter clock-wise,
first sector contains beams for angles  [0, $\frac{2\pi}{N}$).

More formally speaking, assume $\alpha$ is an angle between $x$ axis and
point $(x, y)$ value of $\mathcal{F}(x, y) \in Z$ so that
\begin{equation}
\label{eq:F_def}
\mathcal{F}(x, y)\frac{2\pi}{N} \leq \alpha < (\mathcal{F}(x, y) + 1)\frac{2\pi}{N}
\end{equation}
or, equivalently
$$
\label{eq:F_def_floor}
\mathcal{F}(x, y) = \lfloor N \frac{\alpha}{2\pi} \rfloor
$$
In this document we'll use either two-argument form $\mathcal{F}(x, y)$ when
calclating arcatngent for point $(x, y)$ or single argument form
$\mathcal{F}(\alpha)$ for simplicity - here $\alpha$ is an angle between $x$ axis
and $(x,y)$ point, as written above.

\section{Math background}
\subsection{Reduction problem to angles only in $[0, \frac{\pi}{4})$}

If $\alpha \ge \pi$ then we can compute
\begin{equation}
\mathcal{F}(\alpha - \pi) =
\lfloor \frac{(\alpha - \pi)N}{2\pi} \rfloor =
\lfloor \frac{\alpha N}{2\pi} - \frac{N}{2} \rfloor =
\lfloor \frac{\alpha N}{2\pi} \rfloor - \frac{N}{2} =
\mathcal{F}(\alpha) - \frac{N}{2}
\end{equation}
Here last expression is valid since $\frac{N}{2} \in Z$ by definition of
$N$ suported by the library. Re-ordering last inequality gives us
\begin{equation}
\label{eq:F_mirror_1}
\mathcal{F}(\alpha) = \mathcal{F}(\alpha - \pi) + \frac{N}{2}
\end{equation}
Calculating point $(x', y')$ for angle $\alpha - \pi$ from point $(x, y)$ is
straighforward. We can re-write original point coords as
\begin{equation}
x = R\cos(\alpha); y = R\sin(\alpha)
\end{equation}
where $R$ is a distance from coordinate origin to point $(x, y)$. So
\begin{multline}
\label{eq:F_mirror_1_coords}
x' = R\cos(\alpha - \pi) = R(\cos(\alpha)\cos(\pi) + \sin(\alpha)\sin(\pi)) =
-R\cos(\alpha) = -x \\
y' = R\sin(\alpha - \pi) = R(\sin(\alpha)\cos(\pi) - \cos(\alpha)\sin(\pi)) =
-R\sin(\alpha) = -y
\end{multline}

If $\pi > \alpha \ge \frac{\pi}{2}$ then same as in \eqref{eq:F_mirror_1} we get
\begin{equation}
\label{eq:F_mirror_2}
\mathcal{F}(\alpha) = \mathcal{F}(\alpha - \frac{\pi}{2}) + \frac{N}{4}
\end{equation}
And new coords $(x', y')$
\begin{multline}
\label{eq:F_mirror_2_coords}
x' = R\cos(\alpha - \frac{\pi}{2}) = R(\cos(\alpha)\cos(\frac{\pi}{2}) + \sin(\alpha)\sin(\frac{\pi}{2})) =
R\sin(\alpha) = y \\
y' = R\sin(\alpha - \frac{\pi}{2}) = R(\sin(\alpha)\cos(\frac{\pi}{2}) - \cos(\alpha)\sin(\frac{\pi}{2})) =
-R\cos(\alpha) = -x
\end{multline}

And finally if $\frac{\pi}{2} > \alpha \ge \frac{\pi}{4}$ then same as in \eqref{eq:F_mirror_1} we get
\begin{equation}
\label{eq:F_mirror_3}
\mathcal{F}(\alpha) = \mathcal{F}(\alpha - \frac{\pi}{4}) + \frac{N}{8}
\end{equation}
And new coords $(x', y')$
\begin{multline}
x' = R\cos(\alpha - \frac{\pi}{4}) = R(\cos(\alpha)\cos(\frac{\pi}{4}) + \sin(\alpha)\sin(\frac{\pi}{4})) =
R\frac{\sqrt{2}}{2}(\cos(\alpha) + \sin(\alpha)) = \frac{\sqrt{2}}{2}(x + y)\\
y' = R\sin(\alpha - \frac{\pi}{4}) = R(\sin(\alpha)\cos(\frac{\pi}{4}) - \cos(\alpha)\sin(\frac{\pi}{4})) =
R\frac{\sqrt{2}}{2}(\sin(\alpha) - \cos(\alpha)) = \frac{\sqrt{2}}{2}(y - x)
\end{multline}
Since multiplying both coords to the same scalar does not change angle $\alpha$ we're interested in, we
can use in our calculations.
\begin{multline}
\label{eq:F_mirror_3_coords}
x' = (x + y) \\
y' = (y - x) \\
\end{multline}

\subsection{Computing function for angles in $[0, \frac{\pi}{4})$}
Assume $f = \mathcal{F}(\alpha)$ for $\alpha \in [0, \frac{\pi}{4})$.
On that interval range for f, assuming N is multiple by 8.
\begin{equation}
\label{eq:f_range}
0 \leq f < \frac{N}{8}
\end{equation}
Or, since $f \in Z$
\begin{equation}
\label{eq:f_range_2}
0 \leq f \leq \frac{N}{8} - 1
\end{equation}

Since $\tan$ function monotonically increases on that interval we can write
\begin{equation}
\label{eq:f_def_2}
\tan(f\frac{2\pi}{N}) \leq \tan(\alpha) < \tan((f + 1)\frac{2\pi}{N})
\end{equation}

Let's introduce table T:
\begin{equation}
\label{eq:T_def}
T[k] = \tan(\frac{2\pi k}{N}), k \in Z, k \in [0, \frac{N}{8}]
\end{equation}
Note that $T[k+1] > T[k] \forall  k$ since $\tan$ is monotonically increasing.

\begin{equation}
\label{eq:T_alpha_ineq}
T[f] \le \tan(\alpha) < T[f + 1]
\end{equation}
and our task is quickly find elemend $f$, satisfying this inequality.

Assume we can build index table $J$ of size $S$, so that
\begin{equation}
\label{eq:J_def}
T[J[k] - 1] < \frac{k}{S} \leq T[J[k]],
J[k] \geq 0 \forall k
\end{equation}
We can satisfy this inequality only if $S > Smin$.
This inequality becomes possible when $\frac{1}{S} < \min_{i}(T[i + 1] - T[i])$.
That is evident - since values $\frac{k}{S}$ placed uniformly, in that
case between any $T[i]$ and $T[i + 1]$ will be at least one $\frac{k}{S}$ value.
So
\begin{multline}
\frac{1}{S} < \min_{i}(\tan(\frac{2\pi (i + 1)}{N}) - \tan(\frac{2\pi i}{N})) \\
 =  \min_{i}(\frac{\sin(\frac{2\pi (i + 1)}{N} - \frac{2\pi i}{N})}{\cos(\frac{2\pi (i + 1)}{N})\cos(\frac{2\pi i}{N})})
 =  \min_{i}(\frac{\sin(\frac{2\pi}{N})}{\cos(\frac{2\pi (i + 1)}{N})\cos(\frac{2\pi i}{N})})
\end{multline}
Since $\cos$ funcion is monotonically decreasing on $[0, \frac{\pi}{2})$, maximum value in
denominator achieved when $i = 0$. So we get
\begin{equation}
\label{eq:Smin_value}
\frac{1}{S} < \frac{\sin(\frac{2\pi}{N})}{\cos(\frac{2\pi}{N})} = \tan(\frac{2\pi}{N}),
S > Smin = \frac{1}{\tan(\frac{2\pi}{N})}
\end{equation}

Having table $J$ we can quickly find values of $f$ with it we can prove that
\begin{equation}
\label{eq:J_ineq_prove}
T[J[\lfloor \tan(\alpha) S \rfloor - 1]] \leq T[f] \leq T[J[\lfloor \tan(\alpha) S \rfloor]]
\end{equation}
in that case we need make only one comparison after computing $\tan(\alpha) = \frac{y}{x}$.

First prove left part of \eqref{eq:J_ineq_prove}. From \eqref{eq:J_def} we get
\begin{equation}
T[J[\lfloor \tan(\alpha) S \rfloor - 1]] < \frac{\lfloor \tan(\alpha) S \rfloor}{S} \leq \frac{\tan(\alpha)S}{S} = \tan\alpha
\end{equation}
From this and from \eqref{eq:T_alpha_ineq} we get
\begin{equation}
\label{eq:P1}
T[J[\lfloor \tan(\alpha) S \rfloor - 1]] < T[f + 1]
\end{equation}
Because values in T monotonically increase, and $f \in Z$ from \eqref{eq:P1} we have
\begin{equation}
T[J[\lfloor \tan(\alpha) S \rfloor - 1]] \leq T[f]
\end{equation}
that is the left part of \eqref{eq:J_ineq_prove}, Q.E.D.

Proof of the right part of \eqref{eq:J_ineq_prove}. Assume that it is wrong, that is
\begin{equation}
\label{eq:Wrong_inequality}
T[f] > T[J[\lfloor \tan(\alpha) S \rfloor]]
\end{equation}
From \eqref{eq:T_def} and \eqref{eq:F_def_floor}
\begin{multline}
T[f] = \tan(\frac{2\pi f}{N}) = \tan(\frac{2\pi \lfloor N \frac{\alpha}{2 \pi} \rfloor}{N}) \\
\tan(\frac{2\pi \lfloor N \frac{\alpha}{2 \pi} \rfloor}{N}) <  \tan(\frac{2\pi (\frac{N \alpha}{2 \pi} - 1)}{N}) =
\tan(\alpha - \frac{2\pi}{N})
\end{multline}
Substititute this and \eqref{eq:J_def} to \eqref{eq:Wrong_inequality} and get
\begin{multline}
\tan(\alpha - \frac{2\pi}{N}) > T[f] > T[J[\lfloor \tan(\alpha) S \rfloor]] \ge \frac{\lfloor \tan(\alpha) S \rfloor}{S} \\
\tan(\alpha - \frac{2\pi}{N}) > \frac{\lfloor \tan(\alpha) S \rfloor}{S} > \frac{\tan(\alpha) S - 1}{S} = \tan(\alpha) - \frac{1}{S} \\
\frac{1}{S} > \tan(\alpha) - \tan(\alpha - \frac{2\pi}{N}) = \frac{\sin(\frac{2\pi}{N})}{\cos(\alpha)\cos(\frac{2\pi}{N})} = \\
  \tan(\frac{2\pi}{N}) \frac{1}{\cos(\alpha)} \ge \tan(\frac{2\pi}{N})
\end{multline}
But before in \eqref{eq:Smin_value} we took $\frac{1}{S} < \tan(\frac{2\pi}{N})$. We get to contradiction,
so right part of \eqref{eq:J_ineq_prove} is proven $\forall S > Smin$.

\section{Algorighm}
\begin{enumerate}
\item Build table T as described in \eqref{eq:T_def}.
\item Take $S = \lceil \frac{1}{\tan(\frac{2\pi}{N})} \rceil$.
\item Build table J as described in \eqref{eq:J_def}.
\item For each input point:
\begin{enumerate}
\item For input point $(x, y)$ get corresponding point $(x', y')$, arctangent of
      which lies in $[0, \frac{\pi}{4})$. Also compute required offset for result.
      Use for that consequently equations
      \eqref{eq:F_mirror_1}, \eqref{eq:F_mirror_1_coords},
      \eqref{eq:F_mirror_2}, \eqref{eq:F_mirror_2_coords},
      \eqref{eq:F_mirror_3}, \eqref{eq:F_mirror_3_coords}.
\item Calculate $\tan(\alpha) = \frac{x'}{y'}.$
\item Make initial guess $r = J[\lfloor \tan(\alpha)S \rfloor] - 1$. From
    \eqref{eq:J_ineq_prove} we know that either $f = r$ or $f = r + 1$.
\item Compare $\tan(\alpha)$ with $T[r + 1]$. Accorgind to \eqref{eq:T_alpha_ineq}
    must be $\tan(\alpha) < T[f + 1]$ if this inequality does not hold for $f = r$, we
    must use $f = r + 1$.
\item Get final function result $\mathcal{F}(x, y)$ having computed value of
$f$ for $(x', y')$ and offset for it.
\end{enumerate}
\end{enumerate}
For sake of better CPU cache usage we can put both value of $J[r]$ and $T[r + 1]$
in the same table item.

\end{document}
